\usepackage{fontspec}
\newfontfamily\NotoEmoji{Noto Emoji}

\usepackage{amsmath}
%For \CIRCLE
\usepackage{wasysym}


\usepackage{tikz,pgfplots}
\usetikzlibrary{
	positioning,
	shapes.geometric, shapes.callouts,
	external,
	graphs, graphdrawing,
	trees,
	backgrounds,
	fit,
	3d,
	svg.path,
	decorations.pathreplacing,
	calligraphy}
\usegdlibrary{trees, layered}

%\tikzset{external/system call={lualatex -enable-write18 -halt-on-error -interaction=batchmode -jobname "\image" "\texsource"}}
% \tikzset{external/system call={lualatex -shell-escape -halt-on-error -interaction=batchmode -jobname "\image" "\texsource"}}
% \tikzexternalize[prefix=figures/]

\usepackage{forest}
\useforestlibrary{edges}
% For using forked edge style
\ProvidesForestLibrary{edges}[2016/12/05 v0.1.1]

% % For scaling tikzpictures to a width
% % https://tex.stackexchange.com/questions/481915/how-can-i-scale-tikz-picture-to-full-width
% % https://tex.stackexchange.com/questions/6388/how-to-scale-a-tikzpicture-to-textwidth
% \usepackage{environ}
% \newsavebox{\measure@tikzpicture}
% \NewEnviron{scaletikzpicturetowidth}[1]{
%   \def\tikz@width{#1}
%   \def\tikzscale{1}\begin{lrbox}{\measure@tikzpicture}
%   \BODY
%   \end{lrbox}
%   \pgfmathparse{#1/\wd\measure@tikzpicture}
%   \edef\tikzscale{\pgfmathresult}
%   \BODY
% }

% \newcommand{\inputtikz}[1]{\tikzsetnextfilename{#1}\input{#1.tikz.tex}}

% \usepackage[dvipsnames]{xcolor}
\definecolor{semioturquoise}{HTML}{00A69D}
\definecolor{semiopink}{HTML}{FF344F}
\definecolor{semioblue}{HTML}{334fea}
\definecolor{semioyellow}{HTML}{eacf33}
\definecolor{semioorange}{RGB}{240,117,24}
\definecolor{semiopurple}{HTML}{9d00a5}
\definecolor{semioanthrazite}{RGB}{75,85,100}
\definecolor{semioanthraziteLite}{RGB}{191,216,255}
\definecolor{semioanthraziteDark}{RGB}{37,42,50}

\tikzset{
	semioGeneralTerm/.style={draw,loosely dashed},
	semioTerm/.style={draw,dashed},
	semioSpecificTerm/.style={draw,densely dashed},
	semioGeneralItem/.style={draw,loosely dotted},
	semioItem/.style={draw,dotted},
	semioSpecificItem/.style={draw,densely dotted},
	semioTermText/.style={},
    sobject/.style={circle,node font=\normalsize,minimum size=0,inner sep=0pt},
    sobjectDescription/.style={rectangle,align=left,node font=\tiny},
	attractionEdge/.style={->},  
    port/.style={sloped,anchor=south,node font=\tiny},
	new/.style={semioturquoise,fill=white},
	old/.style={black},
	modified/.style={semioorange},
}

\tikzset{
	semioScheme/.style={draw=black,dotted},
	semioRule/.style={draw=black,dotted},
	ruleCopy/.style={semiopink,->},
	ruleInput/.style={semiopurple,<-},
	ruleOutput/.style={semiopurple,->},
	resolvesTo/.style={semioblue,->},
	semioLayout/.style={draw=black,dotted},
	semioDesign/.style={draw=black,dotted},
	scriptCall/.style={->},
	connection/.style={--}
}
	

\usepackage{ifthen}

%---------------------------------------SYMBOLS---------------------------------------

\newcommand{\emoji}[1]{{\NotoEmoji{#1}}}
% math emoji
\newcommand{\memoji}[1]{{\text{\emoji{#1}}}}

\newcommand{\sTabSpace}{0.15cm}
\newcommand{\sTabs}[1]{\foreach \s in {1,...,#1} {\hspace{\sTabSpace}}}


\def\semioname{Semio}
\def\semioversion{0.2.0}

\newcommand{\sobjectWithDescriptionDescriptionSpace}{0.1cm}

\newcommand{\sUri}{\emoji{🔗}}
\newcommand{\sAttribute}{\emoji{🏷️}}
\newcommand{\attributeEqual}[2]{$\text{\sAttribute}[\memoji{#1}]=#2$}
\newcommand{\attributeAssign}[2]{$\text{\sAttribute}[\memoji{#1}]:#2$}

\newcommand{\sPrototype}{☐}
\newcommand{\sDefinition}{\emoji{🏗️}}
\newcommand{\sPlan}{\emoji{🛠️}}
\newcommand{\parameterEqual}[2]{$\text{\sPlan}[\memoji{#1}]=#2$}
\newcommand{\parameterAssign}[2]{$\text{\sPlan}[\memoji{#1}]:#2$}

\newcommand{\sPort}{\emoji{🪝}}
\newcommand{\TransformationArrow}{\emoji{➡️}}

%---------------------------------------GENERAL PURPOSE FUNCIONS---------------------------------------

%inputs: cardinal direction
\newcommand{\cardianalToDirection}[1] {\ifthenelse{\equal{#1}{north}}{above}{\ifthenelse{\equal{#1}{east}}{right}{\ifthenelse{\equal{#1}{south}}{below}{left}}}}

%inputs: direction
\newcommand{\directionToCardinal}[1] {\ifthenelse{\equal{#1}{above}}{north}{\ifthenelse{\equal{#1}{right}}{east}{\ifthenelse{\equal{#1}{below}}{south}{west}}}}

%inputs: cardinal direction
\newcommand{\invertCardianal}[1] {\ifthenelse{\equal{#1}{north}}{south}{\ifthenelse{\equal{#1}{east}}{west}{\ifthenelse{\equal{#1}{south}}{north}{east}}}}

%inputs: direction
\newcommand{\invertDirection}[1] {\ifthenelse{\equal{#1}{above}}{below}{\ifthenelse{\equal{#1}{right}}{left}{\ifthenelse{\equal{#1}{below}}{above}{right}}}}

%---------------------------------------SPECIAL PURPOSE FUNCIONS---------------------------------------

%inputs: nodeId, uri, prototype, displayId, location attributes
\newcommand{\sobject}[5]{
	\node [sobject,#5] (#1) {$\text{#2}_{\text{#4}}^{\text{#3}}$};}
	
%inputs: nodeId, uri, prototype, displayId, location attributes, content, description attributes
\newcommand{\sobjectWithDescription}[8]{
	\sobject{#1}{#2}{#3}{#4}{#5}
	\node [semioSpecificItem,sobjectDescription] (#1D)
		[rectangle callout,callout absolute pointer={(#1)},
		#6=\sobjectWithDescriptionDescriptionSpace of #1,#8]
		{#7};}

%inputs: attractingId, attractedId
\newcommand{\sAttraction}[2]{
	\draw (#1) [attractionEdge]-- (#2);}

%inputs: attractingId, attractedId, knots, attractedPorts, attractingPorts, id, knotAnchor, attractingAnchor, attractedAnchor, idAnchor
%WARNING: 10th parameter needs to be passed over helper variable: \def\tempVarAAttractionWithPorts{idAnchor}
%Latex only accepts up to 9 parameters...
\newcommand{\sAttractionWithPorts}[9]{
	\draw (#1) [attractionEdge]-- (#2)
		node[port, at start,anchor=#7]{#4} %port attracting
		node[port, midway,anchor=center]{#3} %knots
		node[port, at start,anchor=#9]{#6} %id
		node[port, at end,anchor=#8]{#5};} %port attracted

%inputs: rulePrefix, ruleIcon, xCoordinateLHS, yCoordinateLHS,xCoordinateRHS, yCoordinateRHS,
\newcommand{\sTransformation}[6]{
	\begin{scope}[local bounding box/.expanded=#1lbB]
		\pic (#1L) at (#3,#4)  {#1L};
		\pic (#1R) at (#5,#6)  {#1R};
	\end{scope}
	\node [semioSpecificTerm,fit=(#1lbB)] (#1bB) {};
	\draw (#1LbB.east)[draw=none]--(#1RbB.west) node [midway,label={[anchor=south]north:{#2}}] {\TransformationArrow};}

%https://tex.stackexchange.com/questions/609638/tikz-local-bounding-box-in-nested-scopes
\pgfmathsetmacro\designlevel{0}
\newenvironment{design}[0]
{\pgfmathsetmacro\designlevel{int(\designlevel+1)}
\begin{scope}[local bounding box/.expanded=bounding box \designlevel,sharp corners]}
{\end{scope}\node [semioDesign,fit=(bounding box \designlevel)] (bB) {};}

\newenvironment{designLast}[0]
{\begin{scope}[local bounding box/.expanded=bounding box designLastlevel,sharp corners]}
{\end{scope}\node [semioTerm,fit=(bounding box designLastlevel)] (bB) {};}
		

%https://tex.stackexchange.com/questions/609638/tikz-local-bounding-box-in-nested-scopes
\pgfmathsetmacro\definitionLevel{0}
\newenvironment{definition}[1]
  { % Auxillary variable as workaround to pass argument into custom environment command
    % https://tex.stackexchange.com/questions/380277/custom-environment-with-minipages-causes-illegal-parameter-number-in-definition
    \def\labelNameAuxiallaryVariable{#1}
    \pgfmathsetmacro\definitionLevel{int(\definitionLevel+1)}
  \begin{scope}[local bounding box/.expanded=bounding box \definitionLevel,sharp corners]}
  {\end{scope}
    \node [semioSpecificTerm,fit=(bounding box \definitionLevel),
      label={[semioTermText, anchor=north east]north east:{\labelNameAuxiallaryVariable}}] 
      (bB) {};}

%https://tex.stackexchange.com/questions/609638/tikz-local-bounding-box-in-nested-scopes
\pgfmathsetmacro\transformationSideLevel{0}
\newenvironment{transformationSide}[0]
  {\pgfmathsetmacro\transformationSideLevel{int(\transformationSideLevel+1)}
  \begin{scope}[local bounding box/.expanded=bounding box \transformationSideLevel,sharp corners]}
  {\end{scope}
    \node [semioItem,fit=(bounding box \transformationSideLevel)](bB) {};}

